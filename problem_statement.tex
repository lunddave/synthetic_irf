\documentclass[11pt]{article}

% use packages
\usepackage[utf8]{inputenc}
\usepackage{amsmath}
\usepackage{amsthm}
\usepackage{amsfonts}
\usepackage{amscd}
\usepackage{amssymb}
\usepackage{natbib}
\usepackage{url}
\usepackage[table,xcdraw,usenames]{xcolor}
%\usepackage[usenames]{color}

\usepackage{graphicx}
%\usepackage{mathtools}
\usepackage{enumitem}
\usepackage{authblk}
\usepackage{bm}


\usepackage{hyperref}
\usepackage{caption}
\usepackage{float}
\usepackage[caption = false]{subfig}
\usepackage{tikz}
\usepackage{multirow}
\usepackage[linesnumbered, ruled,vlined]{algorithm2e}
\usepackage{pdflscape}

% margin setup
\usepackage{geometry}
\geometry{margin=0.8in}


% function definition
\newcommand{\R}{\mathbb{R}}
\newcommand{\w}{\textbf{w}}
\newcommand{\x}{\textbf{x}}
\newcommand{\dbf}{\textbf{d}}
\newcommand{\y}{\textbf{y}}
\newcommand{\X}{\textbf{X}}
\newcommand{\Y}{\textbf{Y}}
\newcommand{\L}{\textbf{L}}
\newcommand{\Hist}{\mathcal{H}}
\newcommand{\Prob}{\mathbb{P}}
\def\mbf#1{\mathbf{#1}} % bold but not italic
\def\ind#1{\mathrm{1}(#1)} % indicator function
\newcommand{\simiid}{\stackrel{iid}{\sim}} %[] IID 
\def\where{\text{ where }} % where
\newcommand{\indep}{\perp \!\!\! \perp } % independent symbols
\def\cov#1#2{\mathrm{Cov}(#1, #2)} % covariance 
\def\mrm#1{\mathrm{#1}} % remove math
\newcommand{\reals}{\mathbb{R}} % Real number symbol
\def\t#1{\tilde{#1}} % tilde
\def\normal#1#2{\mathcal{N}(#1,#2)} % normal
\def\mbi#1{\boldsymbol{#1}} % Bold and italic (math bold italic)
\def\v#1{\mbi{#1}} % Vector notation
\def\mc#1{\mathcal{#1}} % mathical
\DeclareMathOperator*{\argmax}{arg\,max} % arg max
\DeclareMathOperator*{\argmin}{arg\,min} % arg min
\def\E{\mathbb{E}} % Expectation symbol
\def\mc#1{\mathcal{#1}}
\def\var#1{\mathrm{Var}(#1)} % Variance symbol
\def\checkmark{\tikz\fill[scale=0.4](0,.35) -- (.25,0) -- (1,.7) -- (.25,.15) -- cycle;} % checkmark
\newcommand\red[1]{{\color{red}#1}}
\def\bs#1{\boldsymbol{#1}}
\def\P{\mathbb{P}}
\def\var{\mathbf{Var}}
\def\naturals{\mathbb{N}}
\def\cp{\overset{p}{\to}}
\def\clt{\overset{\mathcal{L}^2}{\to}}

\setcounter{tocdepth}{4}
\setcounter{secnumdepth}{4}

\newtheorem{corollary}{Corollary}
\newcommand{\ceil}[1]{\lceil #1 \rceil}
\newcommand{\norm}[1]{\left\lVert#1\right\rVert} % A norm with 1 argument
\DeclareMathOperator{\Var}{Var} % Variance symbol

\newtheorem{cor}{Corollary}
\newtheorem{lem}{Lemma}
\newtheorem{thm}{Theorem}
\newtheorem{defn}{Definition}
\newtheorem{prop}{Proposition}
\theoremstyle{definition}
\newtheorem{remark}{Remark}
\hypersetup{
  linkcolor  = blue,
  citecolor  = blue,
  urlcolor   = blue,
  colorlinks = true,
} % color setup

% proof to proposition 
\newenvironment{proof-of-proposition}[1][{}]{\noindent{\bf
    Proof of Proposition {#1}}
  \hspace*{.5em}}{\qed\bigskip\\}
% general proof of corollary
  \newenvironment{proof-of-corollary}[1][{}]{\noindent{\bf
    Proof of Corollary {#1}}
  \hspace*{.5em}}{\qed\bigskip\\}
% general proof of lemma
  \newenvironment{proof-of-lemma}[1][{}]{\noindent{\bf
    Proof of Lemma {#1}}
  \hspace*{.5em}}{\qed\bigskip\\}

\allowdisplaybreaks

\title{Synthetic Impulse Response Functions}
\author{David Lundquist\thanks{davidl11@ilinois.edu} }
\affil{Department of Statistics, University of Illinois at Urbana-Champaign}
\date{November 3rd, 2022}

\begin{document}

\maketitle

\begin{abstract}
We adopt techniques from an inferential procedure known as synthetic control to construct a new impulse response function (IRF) estimator.  Distinct from estimates generated from Wold decomposition or local projections (LP), synthetic IRFs leverage information from the context surrounding a shock with the goals of both reducing risk while also limiting bias.  The method relies upon Wold and/or LP IRF estimates on multivariate time series from a "donor pool".  These estimates in turn are aggregated using distanced-based weighting, a procedure in which the donor multivariate series are judged based on similarity to the target multivariate series.  We also develop a procedure to discount the donor series based on signal-to-noise ratio.  This adjustment supports the "unit-shock" convention used in impulse response function analysis.  Simulations and empirical examples are provided.

\end{abstract}

\section{Introduction}

The technique of distanced-based weighting and synthetic, aggregation methods is naturally suited to IRF and vector autoregressions more generally due to the multivariate setting that comes for free.  There is no need to search for predictive or otherwise informative covariates.  

Challenge: suppose a researcher knew that a countable set of donors $\mathcal{D}$ could be used to maximize fit (loosely defined) in the context of impulse response function estimation.  However, for a particular donor $d_{m}, m\in \mathcal{D}$, the shock at time $t^*$ is not observed and hence its magnitude must be inferred.

\begin{enumerate}
    \item Estimate the shock $\epsilon_{i,t^*}$ for each donor.
    \item If estimate requires parametric assumption, then scale estimates appropriately to satisfy the unit-shock assumption.
    \item Confidence intervals?  Strong assumptions needed?
    \item Need to be very clear about whether this is an inferential or predictive tool.
\end{enumerate}

Ideas

\begin{enumerate}
    \item Can we use the Wold decomposition as ground truth?
    \item Can we used synthetic vol forecasting for Value-at-Risk (VaR)?
\end{enumerate}

\section{The estimand}

In a multivariate time series of dimension $K$, the response $i$th-step-ahead response of component $j$ to a shock in component $k$ at time $t$ is

\[
   \frac{\partial y_{j,t+i}}{\partial w_{k,t}}
\]

where $w_{t}$ is the vector of innovations at time $t$ for some model $y_{t} = f(\mathcal{F}_{t-1}) + w_{t}$.

\section{Estimators}

Estimators for an IRF may come in at least two forms: model-specific and model independent.

\subsection{Unit shock - what does it mean to be scale-free or scale-independent?}


\section{The Special Case of Vector Autoregressions (VAR)}





\bibliographystyle{plainnat}
\bibliography{synthVolForecast}
%\bibliography{../synthetic-prediction-notes}

  
\end{document}


