\documentclass[11pt]{article}

% use packages
\usepackage[utf8]{inputenc}
\usepackage{amsmath}
\usepackage{amsthm}
\usepackage{amsfonts}
\usepackage{amscd}
\usepackage{amssymb}
\usepackage{natbib}
\usepackage{url}
\usepackage[table,xcdraw,usenames]{xcolor}
%\usepackage[usenames]{color}

\usepackage{graphicx}
%\usepackage{mathtools}
\usepackage{enumitem}
\usepackage{authblk}
\usepackage{bm}


\usepackage{hyperref}
\usepackage{caption}
\usepackage{float}
\usepackage[caption = false]{subfig}
\usepackage{tikz}
\usepackage{multirow}
\usepackage[linesnumbered, ruled,vlined]{algorithm2e}
\usepackage{pdflscape}

% margin setup
\usepackage{geometry}
\geometry{margin=0.8in}


% function definition
\newcommand{\R}{\mathbb{R}}
\newcommand{\w}{\textbf{w}}
\newcommand{\x}{\textbf{x}}
\newcommand{\dbf}{\textbf{d}}
\newcommand{\y}{\textbf{y}}
\newcommand{\X}{\textbf{X}}
\newcommand{\Y}{\textbf{Y}}
%\newcommand{\L}{\textbf{L}}
\newcommand{\Hist}{\mathcal{H}}
\newcommand{\Prob}{\mathbb{P}}
\def\mbf#1{\mathbf{#1}} % bold but not italic
\def\ind#1{\mathrm{1}(#1)} % indicator function
\newcommand{\simiid}{\stackrel{iid}{\sim}} %[] IID 
\def\where{\text{ where }} % where
\newcommand{\indep}{\perp \!\!\! \perp } % independent symbols
\def\cov#1#2{\mathrm{Cov}(#1, #2)} % covariance 
\def\mrm#1{\mathrm{#1}} % remove math
\newcommand{\reals}{\mathbb{R}} % Real number symbol
\def\t#1{\tilde{#1}} % tilde
\def\normal#1#2{\mathcal{N}(#1,#2)} % normal
\def\mbi#1{\boldsymbol{#1}} % Bold and italic (math bold italic)
\def\v#1{\mbi{#1}} % Vector notation
\def\mc#1{\mathcal{#1}} % mathical
\DeclareMathOperator*{\argmax}{arg\,max} % arg max
\DeclareMathOperator*{\argmin}{arg\,min} % arg min
\def\E{\mathbb{E}} % Expectation symbol
\def\mc#1{\mathcal{#1}}
\def\var#1{\mathrm{Var}(#1)} % Variance symbol
\def\checkmark{\tikz\fill[scale=0.4](0,.35) -- (.25,0) -- (1,.7) -- (.25,.15) -- cycle;} % checkmark
\newcommand\red[1]{{\color{red}#1}}
\def\bs#1{\boldsymbol{#1}}
\def\P{\mathbb{P}}
\def\var{\mathbf{Var}}
\def\naturals{\mathbb{N}}
\def\cp{\overset{p}{\to}}
\def\clt{\overset{\mathcal{L}^2}{\to}}

\setcounter{tocdepth}{4}
\setcounter{secnumdepth}{4}

\newtheorem{corollary}{Corollary}
\newcommand{\ceil}[1]{\lceil #1 \rceil}
\newcommand{\norm}[1]{\left\lVert#1\right\rVert} % A norm with 1 argument
\DeclareMathOperator{\Var}{Var} % Variance symbol

\newtheorem{cor}{Corollary}
\newtheorem{lem}{Lemma}
\newtheorem{thm}{Theorem}
\newtheorem{defn}{Definition}
\newtheorem{prop}{Proposition}
\theoremstyle{definition}
\newtheorem{remark}{Remark}
\hypersetup{
  linkcolor  = blue,
  citecolor  = blue,
  urlcolor   = blue,
  colorlinks = true,
} % color setup

% proof to proposition 
\newenvironment{proof-of-proposition}[1][{}]{\noindent{\bf
    Proof of Proposition {#1}}
  \hspace*{.5em}}{\qed\bigskip\\}
% general proof of corollary
  \newenvironment{proof-of-corollary}[1][{}]{\noindent{\bf
    Proof of Corollary {#1}}
  \hspace*{.5em}}{\qed\bigskip\\}
% general proof of lemma
  \newenvironment{proof-of-lemma}[1][{}]{\noindent{\bf
    Proof of Lemma {#1}}
  \hspace*{.5em}}{\qed\bigskip\\}

\allowdisplaybreaks

\title{Synthetic Impulse Response Functions}
\author{David Lundquist\thanks{davidl11@ilinois.edu}, Daniel Eck\thanks{dje13@illinois.edu} }
\affil{Department of Statistics, University of Illinois at Urbana-Champaign}
\date{\today}

\begin{document}

\maketitle

\begin{abstract}
We adopt techniques from an inferential tool known as synthetic control to construct a new impulse response function (IRF) estimator for multivariate time series.  Distinct from estimates generated from the Wold Decomposition or local projections (LP), Synthetic IRF (SIRF) leverages information from the context surrounding a shock with the goals of both reducing risk while also limiting bias.  The method relies upon Wold and/or LP IRF estimates on multivariate time series from a donor pool, i.e. a collection of multivariate time series subject to a qualitatively similar shock.  These estimates in turn are aggregated using distanced-based weighting, a procedure in which the donor multivariate series are judged based on similarity to the target multivariate series.  We also develop a procedure to discount the donor series based on signal-to-noise ratio.  This adjustment supports the "unit-shock" convention used in impulse response function analysis.  Simulations and empirical examples are provided.

\end{abstract}

\section{Introduction}

The technique of distanced-based weighting and synthetic aggregation methods is naturally suited to IRF and vector autoregressions more generally due to the multivariate setting that comes for free.  There is not necessarily a need to locate predictive or otherwise informative covariates, although there may be additional covariates that parameterize a shock distribution.

Challenge: suppose a researcher knew that a countable set of donors $\mathcal{D}$, i.e. a convex combination thereof, could be used to estimate the impulse response function of a multivariate time series.  However, for a particular donor $d_{m}, m\in \mathcal{D}$, the shock at time $T_{m}^*$ is not observed and hence its magnitude must be inferred.

\begin{enumerate}
    \item Estimate the shock $\epsilon_{i,T^*+1}$ for each donor.
    \begin{enumerate}
      \item Use estimated residuals
      \item Use fixed effect to estimate shock.  Now, is this counter to VAR analysis?
    \end{enumerate}
    \item If the estimation technique parametric assumption, then we can scale estimates appropriately to satisfy the unit-shock assumption.
    \item Confidence intervals?  Strong assumptions needed?
    \item Need to be very clear about whether this is an inferential or predictive tool.  Or is it both?
\end{enumerate}

\section{Setting}
We model structural shocks as affine functions of prevailing macroeconomic, financial, monetary, and/or technological conditions, with the motivation being that lurking danger is always tied to underlying structural variables.

\section{Questions}

\begin{enumerate}
  \item Do we wait until the shock has already manifested?
  \begin{enumerate}
    \item If so, then it's different from other applications of post-shock forecasting.
    \item If not, then are we combining post-shock forecasting with IRF estimation?
  \end{enumerate}
\end{enumerate}

\subsection{What are we estimating and what are we adjusting?}
\begin{enumerate}
  \item Do we need to estimate a model for the time series under study and then adjust that model?
  \begin{enumerate}
  \item Perhaps that for the TSUS, we can avoid assuming a particular model / method for the IRF and can instead just use dbw on the IRFs of the donors. 
  \end{enumerate}
  \item How do we get IRFs of donors? Either MA representation or local projections?
  \item What is it exactly that we're trying to predict?  The effect of a size-delta shock in series A on series B exactly K time points after the shock has manifested.
  \item If we are willing to dispense with linearity (which seems like a reasonable choice, given that we're employing an approach that doesn't have a problem accommodating it)
\end{enumerate}

\begin{enumerate}
    \item Can we use the Wold Decomposition as ground truth?
    \item read \citet{ho2023averaging}
\end{enumerate}

\section{The estimand}

In a multivariate time series of dimension $K$, the response $i$th-step-ahead response of component $j$ to a shock in component $k$ at time $t$ is

\[
   \frac{\partial y_{j,t+i}}{\partial w_{k,t}}
\]

where $w_{t}$ is the vector of innovations at time $t$ for some model $y_{t} = f(\mathcal{F}_{t-1}) + w_{t}$.

\section{Estimators}

Estimators for an IRF may come in at least two forms: model-specific and model independent.

\subsection{Unit shock - what does it mean to be scale-free or scale-independent?}

\section{The Special Case of Vector Autoregressions (VAR)}

\bibliographystyle{plainnat}
\bibliography{problem_statement}
  
\end{document}


